\section{Introduction}
Wave-particle interaction, especially the nonlinear resonant one is of great importance and have been extensively discussed in a broad spectrum of plasma physics. 
In magnetically confined fusion devices, there have Alfv\'en wave instabilities \cite{chen2016,wang2018} which are associated with mode frequency sweeping and lead to premature ejection of alpha particles that deteriorate plasma confinement \cite{fasoli2007}.
In the planetary magnetosphere \cite{tsurutani1974}, as well as in the laboratory devices \cite{vancompernolle2015,vancompernolle2017a}, people have studied the chorus wave, which is associated with various geophysical activity like relativistic electron precipitation, X-ray microbursts, pulsating and diffuse auroras \cite{kasahara2018,reeves2013,thorne2013}.
It has been demonstrated that the nonlinear resonant interaction between toroidal Alfv\'en eigenmodes and fusion-born alpha particles, the energetic electrons and the whistler-mode wave play the key role in such nonlinear instabilities, and therefore it is greatly concerned to build solid and various theoretical descriptions and numerical models to further discuss those wave-particle interactions.

In general, the wave-particle interactions can be described by the Vlasov equation and self-consistently coupled with the wave evolution. 
In the homogeneous plasmas \cite{lilley2009,breizman2010} or in the nonuniform regime with spatial symmetry \cite{hezaveh2017,hezaveh2020,hezaveh2021}, the involved wave can be treated as standing wave with stationary fixed wavenumber or mode number, and the interactions are localized without any spatial dependence.
Thus the simulation is relatively straightforward due to the system's inherent periodicity. 
It only needs to consider the momentum space in one local spatial volume, which is essentially a single scale wave-particle interaction problem and favorable to solve numerically.

%!!! dimension all fast
However, when dealing with inhomogeneous cases where the periodicity falls short, things becomes complicated.
The nonuniform spatial dependence brings extra dimensions on both the wave evolution and the fine structures of resonance particle phase space.
The interaction have to be considered on the whole domain instead of in a single spatial volume.
Although, techniques like WKB approximation \cite{wkb} or slowly varying envelope approximation \cite{svap} can be applied to simplify the wave calculation, it is hard to find a suitable approach to split the scales of the resonant particles due to intricate wave-particle interactions along the inhomogeneous magnetic field.
Therefore, the fast varying temporspatial scales are involved in all dimensions of the resonant particle phase space which brings tremendous workload than that in the homogeneous plasma settings.
Also, it becomes clear that traditional methods lack the required precision to describe the nuanced behavior of particles subjected to the nonlinearity imposed by inhomogeneous magnetic fields and chirping waves.

%!!!!! main idea, obtain theta J and decouples with the xi Omega domain
To address these shortcomings, we have proposed a novel theoretical framework \cite{} that  describes the physics on the reference frames moving in the local resonant velocity.
We transcend the stationary phase approximation \cite{spa1,spa2}, and expand the wave phase about the local resonance center, which decouples the motion scales. 
We compose a new Hamiltonian  represented in the new canonical coordinates and momentum. The Hamiltonian can be split into two subspaces. The one is $\xi,~\Omega$ space, corresponding to the fast-varying wave particle interaction terms. The other is $\vartheta,~\mathcal{J}$ space, corresponding to the slowly-varying one seen by the resonance frame moving along the resonance trajectory.
Thus, the fast varying scales in the whole dimensions are now reduced by half, and the fast-varying wave particle interactions in phase space $\xi,~\Omega$ are able to be studied with a relaxed resolution in $\vartheta, \mathcal{J}$ domain.
%!!!! Now... fast scale is not the core, the separated theta and J is the core of this method, it allow us to simulate with large scale
The particle slowly varying scale corresponds to the characteristic length of background plasma inhomogeneity, and the dynamics in $\xi,~\Omega$ subspace can be treated as quasi-periodical in $\xi$ dimension, same as that in the homogeneous plasma. 
Base on previous analysis, we write the corresponding Vlasov equation for the resonance particle distribution function $f(t,s_i,\vartheta,\mathcal{J},\xi,\Omega)$. The equation is readily constructed as an advective form of two separated Hamiltonian flows and the advective term of reframe moving along resonance trajectories.
\begin{equation}\label{eq.Vlasov}
    \frac{\partial f}{\partial t}+ \frac{d s_{i}}{d t} \frac{\partial f}{\partial s_{i}} + \left[f, H\right]_{\vartheta,\mathcal{J}} +  \left[f, H\right]_{\xi,\Omega} = 0~,
\end{equation}
where the bracket is the canonical Poisson bracket.
%For such generallized Vlasov form, it can be solved by 

The above Vlasov equation consists of two separated motions, and we naturally solve it using hybrid method individually. 
Here we consider a delta $f$ Vlasov solver based on an elaborated hybrid Lagrangian-Eulerian method for solving such system \cite{shiroto2022} for such general form.
The Vlasov system is coupled to the slowly varying wave envelope equation in the resonance frame. 
%detailed steps and benifits
The whole system is solved separately by 4 steps.
Firstly, we apply a conservative form Interpolated Differential Operator (IDO-CF)  method to the Eulerian grid to solve the evolution in $\xi,~\Omega$ domain, it provides a high resolution for recognizing the formation and evolution of resonant structures due to the fast varying wave-particle interaction. 
Secondly, the perturbed current is then integrated from the perturbed distribution function with local equilibrium quantities.
Thirdly, the trajectory of Lagrangian markers sampled in slowly varying domain are solved by Runge-Kutta (RK) method, and the perturbed distribution is updated on the new maker coordinates.
Note that, an adaptive time step RK solver is applied to the Eulerian part to link the gap of the time step in $\xi,\Omega$ and $s,\vartheta,\mathcal{J}$ domain.
Finally, after finished the particle solver, we couple the current from the last procedure to the wave equation.
We derive the first and second order wave equation form and solved by RK and implicit upwind scheme.

%as to the wave, the key is coherent form and determins the pertubation H 
%and we only consider the chorus
%From the Ampere's law, we can obtain various forms of wave equation regarding the wave envelope. 
We use a 1D whistler wave frequency chirping problem in the Earth magnetosphere as benchmark.
The results give the linear growth rate, which is in good agreement with linear theoretic predictions.
We also show the trapped particle phase space hole and compared with that given from the theoretic result, which further validate our simulation results at nonlinear stage.
%benchmark
By incorporating advanced numerical techniques and a more comprehensive understanding of the magnetic field variations, we endeavor to uncover the elusive fine structure of resonant particles. This endeavor holds the potential to revolutionize our comprehension of the intricate interplay between waves and particles within inhomogeneous magnetic fields.

%simulation details
We organized our paper as follows. In section~\ref{sec:vlasov}, we present the HLE theory for our scale separated Vlasov system and introduce the IDO-CF numerical scheme. In section~\ref{sec:wave}, we present the details of the numerical modeling on the wave slowly varying envelope. Section~\ref{sec:code} provides a detailed description on our code design and implementations. Finally, the conclusion is provided in section~\ref{sec:end}.

% Before introducing our numerical scheme, we first show the normalization for the equations. 
% The original equations in our previous studies are written in Gaussian unit.
% %The normalization for the next numerical scheme is given in Tab. (\ref{tab.units}).
%\begin{table}\label{tab.units}
%    \centering
%    \caption{\newline The basic normalization units.\newline}
%    \begin{tabular}{llll}
%    \hline
%     Mass  & Charge & velocity & Time \\
%     \hline
%     $m_e$ & $e$ & $c$ & $\omega_{pe}^{-1}$ \\
%     \hline
%    \end{tabular}\\
%\end{table}
% Here we use electron mass $m_e$, electron charge $e$, speed of light $c$ and background electron plasma frequency $\omega_{pe}$ to normalize mass, charge, velocity and time. Other units for quantities like momentum $\mathcal{J}$ and vector potential $A$ can be derived from the related governing equations as:
% \begin{equation}
%     [A]: \frac{m_e c^2}{e}, \quad[\mathcal{J}]: \frac{m_e c^2}{\omega_{pe}}~.
% \end{equation}
% The physical constants are now eliminated in the normalized equations.