%\documentclass{elsarticle}
%%
\documentclass[times,12pt,3p,longtitle]{elsarticle}
\usepackage{makecell}

%%
\usepackage{hyperref}
\hypersetup{
    colorlinks=true,
    linkcolor=blue,
    filecolor=magenta,      
    urlcolor=cyan,
    pdftitle={Overleaf Example},
    pdfpagemode=FullScreen,
    }

%% Stylefile to load JCOMP template
%% The amssymb package provides various useful mathematical symbols
\usepackage{amssymb}
\usepackage{latexsym}
%\journal{Journal of Computational Physics}
\usepackage{amsmath}
\begin{document}

\begin{frontmatter}
\title{A hybrid Eulerian-Lagrangian Vlasov method for nonlinear wave-particle interaction in weakly inhomogeneous magnetic field}
\author[1]{Jiangshan Zheng}

\author[2]{Ge Wang}
\author[1]{Bo Li}
\address[1]{School of Physics, Beihang University, Beijing, Beijing 100191, China}
\address[2]{Institute for Fusion Studies, The University of Texas, Austin, Texas, 78712, USA}
\begin{abstract}
  We present a hybrid Eulerian-Lagrangian Vlasov solver for nonlinear resonant wave-particle interactions problems in weakly inhomogeneous magnetic field.
  The governing Vlasov equation is derived from a recently proposed resonance tracking Hamiltonian method. 
  The method tracks the dynamics of deeply trapped resonant particles in a coherent wave. 
  It gives the evolution of the distribution function with a scale separated Hamiltonian that containing the slowly-varying motion about the resonant frame of reference and fast-varying coherent interactions.
  The hybrid method solves the slowly-varying phase space dynamics by Lagrangian method along the resonant trajectory, while simultaneously manages the fast-varying phase-space evolution in Eulerian grid with an adaptive time step interpolated differential operator scheme.
  We apply our method to study the frequency chirping problem of whistler mode chorus wave in the Earth magnetosphere and successfully reproduce the chirping chorus wave. 
  The results in well agree with previous research. 
  With a focus on nonlinear resonant wave-particle interaction in the weakly inhomogeneous regime, our approach can give high resolution for the fast varying phase space at low computational cost.  
  It could provide additional insights of the wave instabilities and particle energy transfer compared to the conventional Vlasov and particle-in-cell methods.
\end{abstract}
%xtend the periodical condition in the resonance frame
\end{frontmatter}
%\linenumbers

%main text
\clearpage

%intro the wave and the vlasov
\section{Introduction}
Wave-particle interaction, especially the nonlinear resonant interaction, is of great importance and have been extensively discussed in a broad spectrum of plasma physics. 
In magnetically confined fusion devices, there have Alfv\'en wave instabilities \cite{chen2016,wang2018,wang2012,wang2012a} which are associated with mode frequency sweeping and lead to premature ejection of alpha particles that deteriorate plasma confinement \cite{fasoli2007}.
In the planetary magnetosphere \cite{tsurutani1974}, as well as in the laboratory devices \cite{vancompernolle2015,vancompernolle2017a}, researchers have studied the chorus wave, which is associated with various geophysical activity like relativistic electron precipitation, X-ray microbursts, pulsating and diffuse auroras \cite{kasahara2018,reeves2013,thorne2013}.
It has been demonstrated that the nonlinear resonant interaction between toroidal Alfv\'en eigenmodes and fusion-born alpha particles, the energetic electrons and the whistler-mode wave play the key role in these nonlinear instabilities.
Numerical models such as particle-in-cell \cite{chen2015a,xiao2013variational,xiao2020explicit,Xiao2015,xiao2018structure}, and ray tracing \cite{xie2022} can be applied to study these wave-particle interactions and wave propagation. 
However, more generally, the wave-particle interactions can be described by the Vlasov equation self-consistently coupled with the wave evolution.
The corresponding direct Vlasov solver provides higher resolution to the original system. 

For the homogeneous plasmas \cite{lilley2009,breizman2010} or in the nonuniform regime with spatial symmetry \cite{hezaveh2017,hezaveh2020,hezaveh2021}, such consideration is relatively simple.
The involved wave can be treated as standing wave with stationary fixed wavenumber or mode number. 
In such cases, the interactions are confined to the periodical localized regions. Consequently, kinetic simulations in these scenarios become notably straightforward, benefiting from the inherent periodicity of the system. These simulations primarily entail the examination of momentum space within a local spatial volume, effectively addressing a single-scale wave-particle interaction problem, thus rendering them highly amenable to numerical solutions.
However, when dealing with inhomogeneous scenarios where periodicity no longer holds, the complexity significantly escalates. Even weak inhomogeneity can break the periodicity and significantly modify the nonlinear wave-particle interaction.
The nonuniform spatial dependence brings extra dimensions on both the wave evolution and the fine structures of resonance particle phase space.
The interaction have to be considered on the whole domain instead of in a single spatial volume.
Although, techniques like WKB approximation \cite{wkb} or slowly varying envelope approximation \cite{svap} can be applied to simplify the wave calculation, it is challenging to find a suitable approach to split the scales of the resonant particles due to intricate wave-particle interactions along the inhomogeneous magnetic field.
Consequently, the rapidly changing temporal and spatial scales come into play across all dimensions of the resonant particle phase space, resulting in a significantly heavier computational workload than in homogeneous plasma settings.
Also, it becomes clear that traditional methods lack the required precision to describe the nuanced behavior of particles subjected to the nonlinearity imposed by inhomogeneous magnetic fields and chirping waves.

 To address these limitations, %people make efforts 
it is critical
 to decouple the multiple scales of motion to get a reduced description of the system. 
% For the whistler wave packet propagating in a  weakly inhomogeneous magnetic field, Karpman et al. \cite{karpman1974} have proposed a reduced Vlasov theory in which a new integral of motion for the resonant particle is obtained, and therefore degenerating the original system.
% In our recent work , we have proposed a novel theoretical framework that redefines the physics within
%We transcend the stationary phase approximation \cite{spa1,spa2}, and expand 
Recently, a Hamiltonian formulation 
 using the canonical coordinates and momenta
is developed in the reference frames moving at local resonant velocities \cite{zheng2023a}.
The wave phase is expanded about the local resonance center, which effectively decouples  the multiple scales.
The Hamiltonian consists of 
%can be divided into two distinct subspaces, which is similar to Karpman's work but considers self-consistently the variation of both wave number and frequency The one is $\xi,~\Omega$ space, corresponding to 
the fast-varying wave particle interaction terms
and 
%The other is $\vartheta,~\mathcal{J}$ space, corresponding to 
the slowly-varying terms in the resonance frame moving along the resonance trajectory. 
The particle slowly varying scale corresponds to the characteristic length of background plasma inhomogeneity, and the angle variable in the fast varying dynamics %in $\xi,~\Omega$ subspace 
can be treated as quasi-periodic,
% in $\xi$ dimension, 
similar to the treatment in a homogeneous plasma.
% Base on previous analysis, we formulate the corresponding Vlasov equation for the resonance particle distribution function $f(t,s_i,\vartheta,\mathcal{J},\xi,\Omega)$. The equation is readily constructed as an advective form of two separated Hamiltonian flows and the advective term of reference frame moving along resonance trajectories.
% \begin{equation}\label{eq.Vlasov}
%     \frac{\partial f}{\partial t}+ \frac{d s_{i}}{d t} \frac{\partial f}{\partial s_{i}} + \left[f, H\right]_{\vartheta,\mathcal{J}} +  \left[f, H\right]_{\xi,\Omega} = 0~,
% \end{equation}
% where the bracket is the canonical Poisson bracket. 
%The Vlasov equation consists of two separated motions, each of which is solved individually using a hybrid method.
Here we consider a delta $f$ Vlasov solver based on a hybrid Eulerian-Lagrangian (HEL) method  \cite{shiroto2022} for solving the scale separated Vlasov system.
%detailed steps and benifits
%In our numerical scheme, the entire system is solved through a series of four distinct steps.
In the first step,  a conservative form Interpolated Differential Operator (IDO)  method is applied to the Eulerian grid 
and  an adaptive time step Runge-Kutta (RK)  solver is used to solve the  fast varying dynamics in  phase space. 
This provides high resolution simulation of 
the formation and evolution of resonant structures arising from the rapidly changing wave-particle interactions.
%, allowing us to identify and track 
The perturbed current is then integrated from the perturbed distribution function with local equilibrium quantities.
In the second step, the  Lagrangian markers are sampled in slowly varying domain 
and 
the trajectory of markers
are solved by the RK method.
Subsequently,  the perturbed distribution is updated on the new marker coordinates. 
%It's worth noting that we employ an adaptive time step RK solver in the Eulerian part to bridge the gap in time steps between the $\xi, \Omega$ and $s, \vartheta, \mathcal{J}$ domains.
Finally, after completing the particle solver, 
the Vlasov system is coupled to the wave equation in the resonance frame through the perturbed current.
% to the wave equation and
We evolve the  slowly varying wave envelope 
to the next time step
with the second-order  wave equation
using the RK method.
The first-order wave equation is also solved with 
 an implicit upwind scheme.
The nonlinear resonant interaction between frequency chirping  chorus wave and energetic electrons in the Earth's magnetosphere is used as a benchmark for our simulation scheme.
%The results obtained provide the linear growth rate, which aligns closely with theoretical predictions, validating the accuracy of our simulations.
%Additionally, we present the presence of trapped particle phase space holes and compare them with the theoretical results, further affirming the robustness of our simulation, even at the nonlinear stage.
%benchmark
%By incorporating advanced numerical techniques and a more comprehensive understanding of the magnetic field variations, we endeavor to uncover the elusive fine structure of resonant particles. This endeavor holds the potential to revolutionize our comprehension of the intricate interplay between waves and particles within inhomogeneous magnetic fields.

%simulation details
The paper is  organized  as follows. 
 In section~\ref{sec:vlasov}, we present 
 the HEL scheme for the scale separated Vlasov system.
 % and introduce the IDO-CF numerical scheme. 
 In section~\ref{sec:wave}, we present  the numerical method for solving the slowly varying  wave envelope. Section~\ref{sec:code} provides a detailed benchmark of the simulation code. Finally, the conclusion is provided in section~\ref{sec:end}.



%our numerical scheme begin with the solver of the following deltaf Vlasov system
%explain the form
%\input{cpc/theory}
\section{The Hybrid Eulerian-Lagrangian Solver for the Vlasov system}
\label{sec:vlasov}
Consider the  delta $f$ form of Vlasov system in equation (\ref{eq.Vlasov}). The evolution of  perturbed distribution function $\delta f(\vartheta,\mathcal{J};\xi,\Omega;s,t)$ is
\begin{equation}\label{eq.deltaf}
    \frac{\partial \delta f}{\partial t}+ \frac{d s_{}}{d t} \frac{\partial \delta f}{\partial s_{}} + \left[\delta f, H\right]_{\vartheta,\mathcal{J}} +  \left[\delta f, H\right]_{\xi,\Omega} = \mathcal{S}~,
\end{equation}
where the   Poisson brackets are defined as
\begin{equation}
    [f,~g]_{x,y} = \frac{\partial f}{\partial x}\frac{\partial g}{\partial y}-\frac{\partial f}{\partial y}\frac{\partial g}{\partial x}~.
\end{equation}
Here $\xi,\Omega$ and $\vartheta,\mathcal{J}$ are the canonical variables corresponding to the fast  and slowly varying scales, respectively.
The source term $\mathcal{S}= -\left[f_0, \delta H\right]_{\vartheta, \mathcal{J}} - \left[f_0, \delta H\right]_{\xi, \Omega}$ 
where $f_0$ is the equilibrium distribution function and $\delta H$ denotes the perturbed Hamiltonian due to the resonant wave-particle interactions.
Note that the derivatives of $\delta H$ with respect to the slowly varying coordinates $\vartheta$ and $\mathcal{J}$ can be neglected due to the separation  of perturbation and equilibrium scales
and $\partial f_0/\partial \xi=0$ 
%is usually vanished 
since the equilibrium distribution function does not generally depend on the fast varying angle coordinate $\xi$.
Thus, the source term can be simplified as 
\begin{equation}
     \mathcal{S} = \frac{\partial \delta H}{\partial \xi}\frac{\partial f_0}{\partial \Omega}.
     %\left[f_0, \delta H\right]_{\vartheta, \mathcal{J}} + \left[f_0, \delta H\right]_{\xi, \Omega}
\end{equation}
%detailed physical meanning is not need, just clarify the fast and slow scale
%Note that, follows our
In the Hamiltonian theory for the chorus wave frequency chirping, 
the variation of the angle coordinate $\vartheta$  is proportional to the background inhomogeneity 
%Thus,  the variation of  $\vartheta$ is  small for the weakly inhomogeneous case.
and
 the dynamics of $\mathcal{J}$, $\Omega$ and $\xi$ 
 does not  depend on  $\vartheta$.  
%Since the system is $\vartheta$-free, 
Therefore we neglect the term $ \dot{\vartheta} $ in the Vlasov equation and the  Poisson bracket becomes
%evolution of the  $\vartheta$ with 
\begin{equation}
\left[\delta f, H\right]_{\vartheta,\mathcal{J}}\simeq 
 -\dot{\mathcal{J}} \frac{\partial \delta f}{\partial \mathcal{J}}      
\end{equation}
where the dot denotes the time derivative.

We now apply the hybrid method to the Vlasov system where the fast and slowly varying scales have been separated in the  Hamiltonian theory.
%But unlike the manipulations applied in a conservative HLE method \cite{shiroto2022} which solves the advection equations of shape functions unlike particle-in-cell method on configuration domain and model the velocity by discrete super-particles, 
We model the fast varying  phase space $\xi,\Omega$ by Eulerian method and model the slowly varying coordinate
$\mathcal{J}$ 
and the resonance frame  coordinate $s$ through Lagrangian method.
The 
%Klimontovich description-like 
distribution function is written as \cite{shiroto2022}
\begin{equation}
    \delta f(\xi,\Omega,\mathcal{J},s,t) = \sum_{k,l} g_{k,l}(t,\xi,\Omega)\delta(s-s_k(t),\mathcal{J}-\mathcal{J}_l)~,
\end{equation}
where $g_{k,l}(t,\xi,\Omega)$ is the distribution function in $\xi,\Omega$ space with $k$ and $l$ the index of the Lagrangian marker.
% the  delta function represents the markers in  space $s_k$, and $\mathcal{J}_l$ with .
Then the evolution equation for $g_{k,l}(t,\xi,\Omega)$ for each marker labeled by $k$ and $l$ is
%of the HLE method
\begin{equation}\label{eq.Euler}
\frac{\partial g}{\partial t} + \left[g,H\right]_{\xi,\Omega} = \mathcal{S}~.
\end{equation}
Here we omit the index $k$ and $l$ for $g$ for convenience.
For the Lagrangian step, the marker is required to move with the resonance frame on the spatial domain   
\begin{equation}\label{eq.resonance}
        \dot{s}_k = v_r(s_k(t))~,
\end{equation}
where $v_r$ is  the  resonant velocity.
 %Since the dynamics of momentum $\mathcal{J}$ is fully decoupled with $\xi,~\Omega$,
The motion equation for slowly varying coordinate $\mathcal{J}$ is
% \begin{equation}
%     \begin{aligned}
%         \dot{\vartheta} &= \frac{\iint \mathcal{F}_i(t,\xi,\Omega)\left[\vartheta,H\right]_{\vartheta,\mathcal{J}} \mathrm{d}\xi\mathrm{d}\Omega}{\iint \mathcal{F}_i(t,\xi,\Omega)\mathrm{d}\xi\mathrm{d}\Omega}~,\\
%         \dot{\mathcal{J}} &= \frac{\iint \mathcal{F}_i(t,\xi,\Omega)\left[\mathcal{J},H\right]_{\vartheta,\mathcal{J}} \mathrm{d}\xi\mathrm{d}\Omega}{\iint \mathcal{F}_i(t,\xi,\Omega)\mathrm{d}\xi\mathrm{d}\Omega}~.\\
%     \end{aligned}
% \end{equation}
% Note that, follows our Hamiltonian theory \cite{}, the variation of $\vartheta$ coordinate is proportional to the background inhomogeneity. Thus, for the weakly inhomogeneous case, the variation of the angle coordinate $\vartheta$ is considerably small.
% Besides, the dynamics of $J$, $\Omega$ and $\xi$ are also independent on $\vartheta$, i.e., the system is $\vartheta$-free, and we can neglect the evolution of $\vartheta$. 
% Moreover, the dynamics of momentum $\mathcal{J}$ is fully decoupled with $\xi,~\Omega$ dimension, thus the integration over the fast varying dimension can be eliminated, and the force equations in our simulation scheme are
\begin{equation}
    \begin{aligned}\label{eq.Lagrangian}
        \dot{\mathcal{J}} &= \left[\mathcal{J},H\right]_{\vartheta,\mathcal{J}}~.
    \end{aligned}
\end{equation}

\subsection{The Eulerian step}
Now we solve the fast-varying phase space dynamics for each Lagrangian at $s_k,\mathcal{J}_l$.
For the numerical treatment, the Vlasov equation (\ref{eq.Euler}) is expressed as
\begin{equation}\label{eq.Euler2}
    \frac{\partial g}{\partial t} + m \frac{\partial  g}{\partial \xi} - n \frac{\partial  g}{\partial \Omega}= \mathcal{S}~,
\end{equation}
where 
%$g \equiv \mathcal{F}_i$ and 
%we use $m$, $n$ to represent the derivatives of the Hamiltonian
\begin{equation}
        m(\xi,\Omega) = \frac{\partial H}{\partial \Omega},~ n(\xi,\Omega) = \frac{\partial H}{\partial \xi}~.
\end{equation}
%The partial differential equation is a typical form of , and 
To achieve high-order accuracy, we apply the IDO method \cite{imadera2009} to solve the 2D Vlasov equation in $\xi,\Omega$ domain.
The IDO scheme applies the polynomial as local interpolation functions to replace the distributions along each coordinate in phase space. 
%The differential and integral in the advection parts of the Vlasov equation can be then analytically calculated. 
Here
%, we present a second-order scheme to demonstrate the basic routines. 
a second-order  polynomial is used as interpolation stencil for a function $g(x)$ from $x_i$ to $x_{i+1}$,
\begin{equation}\label{eq.intg}
    G(x,g_{i},g_{i+1},\sigma_{i+\frac{1}{2}}) = a\left(x-x_i\right)^2+b\left(x-x_i\right)+c~,
\end{equation}
where $g_i \equiv g(x_i)$, $g_{i+1} \equiv g(x_{i+1})$ are function values on the grid, and $\sigma_{i+\frac{1}{2}}$ is the cell integral value 
\begin{equation}
    \sigma_{i+\frac{1}{2}} \equiv \int_{x_i}^{x_{i+1}} g(x)~\mathrm{d}x~,
\end{equation}
and the coefficients are determined and from the grid value and the line integral over the cell,
\begin{equation}
    \begin{aligned}
        a & =\frac{3\left(g_i+g_{i+1}\right)}{\Delta x^2}-\frac{6 \sigma_{i+\frac{1}{2}}}{\Delta x^3}~, \\
        b & =-\frac{2\left(2 g_i+g_{i+1}\right)}{\Delta x}+\frac{6 \sigma_{i+\frac{1}{2}}}{\Delta x^2}~, \\
        c & = g_i~.
    \end{aligned}
\end{equation}

For equation (\ref{eq.Euler2}), the derivatives of $g$ with respect to $\xi$ and $\Omega$ are numerically represented by the interpolating stencil introduced above. 
The discretized form of the Vlasov equation becomes,
\begin{equation}\label{eq.disV}
    \begin{aligned}
    \left.\frac{\partial g}{\partial t}\right|_{i,j}   & =  - m_{i,j} \left.\frac{\partial}{\partial \xi} G\left(\xi;g_{i,j},g_{i+1,j},\rho_{i+\frac{1}{2},j}\right)\right|_{i,j} 
    \\
    & + n_{i,j}  \left.\frac{\partial}{\partial \Omega}G\left(\Omega;g_{i,j},g_{i,j+1},\kappa_{i,j+\frac{1}{2}}\right)\right|_{i,j}
    \\
    & +  \mathcal{S}_{i,j}~,
    \end{aligned}
\end{equation}
where $i$ and $j$ are the grid index for $\xi$ and $\Omega$, respectively. 
 %$g_\xi$ and $ g_\Omega$ denote the derivatives along $\xi$ and $\Omega$ directions, 
%The stencil is applied along both $\xi$ and $\Omega$ dimensions. 
The grid and the sampling points are demonstrated in Fig.~\ref{fig.grids}.
\begin{figure}[htbp]
    \centering
    \includegraphics[scale=0.3]{cpc_img/IDO.pdf}
    \caption{The interpolating functions and cell integrated values in $\xi,\Omega$ domain.}
    \label{fig.grids}
\end{figure}
%middle step 1, construct f(xi) and f(Omega)
%Firstly, we construct interpolation function according to the stencil to replace $f$ along $\xi$ and $\Omega$ directions and calculate the derivatives on the $i,j$ grid as
%Applying the interpolation function $G$ defined in Eq.~(\ref{eq.intg})  along $\xi$ and $\Omega$  yields the derivatives
% Then the derivatives are given by 
%\begin{equation}
%    \begin{aligned}
%    \frac{\partial g_{i,j}}{\partial \xi}
%         \simeq \left. \frac{\partial}{\partial \xi} G\left(\xi;g_{i,j},g_{i+1,j},\rho_{i+\frac{1}{2},j}\right)\right|_{i,j}~, \\
%         \frac{\partial g_{i,j}}{\partial \Omega}
% \simeq \left.\frac{\partial}{\partial \Omega}G\left(\Omega;g_{i,j},g_{i,j+1},\kappa_{i,j+\frac{1}{2}}\right)\right|_{i,j}~,
%    \end{aligned}
%\end{equation}

The interpolation relies on the line integrated values $\rho_{i+\frac{1}{2},j}=\int_{\xi_i}^{\xi_{i+1}}g_{j}(\xi)\mathrm{d}\xi$ and $\kappa_{i,j+\frac{1}{2}}=\int_{\Omega_j}^{\Omega_{j+1}}g_{i}(\Omega)\mathrm{d}\Omega$.
%The calculation at this step relies upon $\rho_{i+\frac{1}{2},j}$ and $\kappa_{i,j+\frac{1}{2}}$, which are the line integrated value of $\mathcal{F}$ 
To solve the integral value $\rho$ and $\kappa$, we consider the Eq.~(\ref{eq.Euler2}) on given $j$ and $i$ grid respectively, and integrate it along $\xi$ and $\Omega$ over the grid interval, which yields the time evolution of the integral values
%, accroding to the definition in Eq.~(\ref{eq.intg}). The time evolution of the integral value is derived from the Vlasov equation,
\begin{equation}\label{eq.line}
    \begin{aligned}
    \left.\frac{\partial \rho}{\partial t}\right|_{i+\frac{1}{2},j} &= \int_{\xi_i}^{\xi_{i+1}}[ - m_j(\xi) \left.\frac{\partial g}{\partial \xi}\right|_{j}(\xi) + n_j(\xi) \left.\frac{\partial g}{\partial \Omega}\right|_{j}(\xi) + \mathcal{S}_j(\xi)]~\mathrm{d} \xi
    \\
    \left.\frac{\partial \kappa}{\partial t}\right|_{i,j+\frac{1}{2}}
    &= \int_{\Omega_j}^{\Omega_{j+1}}[ - m_i(\Omega) \left.\frac{\partial g}{\partial \xi}\right|_{i}(\Omega) + n_i(\Omega) \left.\frac{\partial g}{\partial \Omega}\right|_{i}(\Omega) + \mathcal{S}_i(\Omega)]~\mathrm{d} \Omega
    \end{aligned}
\end{equation}
%Then, enters step 2, solve Eq.~(\ref{eq.line})
%In the above integrals on the right-hand-side, 
We apply the third-order central interpolation scheme to approximate the functions $m$ and $n$ along the $\xi$ and $\Omega$ dimension and use the interpolating stencil again to approximate the derivatives, 
\begin{equation}
    \begin{aligned}
       \left.\frac{\partial g}{\partial \xi}\right|_{j}(\xi)
        &\simeq G\left(\xi;\left.\frac{\partial g}{\partial \xi}\right|_{i,j},\left.\frac{\partial g}{\partial \xi}\right|_{i+1,j},g_{i+1,j}-g_{i,j}\right)
        \\
        \left.\frac{\partial g}{\partial \Omega}\right|_{j}(\xi)
         &\simeq G\left(\xi;\left.\frac{\partial g}{\partial \Omega}\right|_{i,j},\left.\frac{\partial g}{\partial \Omega}\right|_{i+1,j},    \left.\frac{\partial \rho}{\partial \Omega}\right|_{i+\frac{1}{2},j}\right)
         \\
         \left.\frac{\partial g}{\partial \Omega}\right|_{i}(\Omega) 
     &\simeq G\left(\Omega;\left.\frac{\partial g}{\partial \Omega}\right|_{i,j},\left.\frac{\partial g}{\partial \Omega}\right|_{i,j+1},g_{i,j+1}-g_{i,j}\right)\\
         \left.\frac{\partial g}{\partial \xi}\right|_{i}(\Omega) 
  &\simeq G\left(\Omega;\left.\frac{\partial g}{\partial \xi}\right|_{i,j},\left.\frac{\partial g}{\partial \xi}\right|_{i,j+1},\left.\frac{\partial \kappa}{\partial \xi}\right|_{i,j+\frac{1}{2}}\right)
    \end{aligned}
\end{equation}
%The interpolations bring new dependence on intergral value $\rho_{\Omega;i+\frac{1}{2},j}$ and $\kappa_{\xi;i,j+\frac{1}{2}}$. Therefore, 
We apply additional interpolation functions to approximate 
%$\rho$ along $\Omega$ and $\kappa$ along $\xi$ to calculate its 
the derivatives
\begin{equation}
    \begin{aligned}
    \left.\frac{\partial \rho}{\partial \Omega}\right|_{i+\frac{1}{2},j}
    &\simeq \left.\frac{\partial}{\partial \Omega} G(\Omega;\rho_{i+\frac{1}{2},j},\rho_{i+\frac{1}{2},j+1}, M_{i+\frac{1}{2},j+\frac{1}{2}})\right|_{i+\frac{1}{2},j}~, \\
    \left.\frac{\partial \kappa}{\partial \xi}\right|_{i,j+\frac{1}{2}}
 &\simeq \left.\frac{\partial}{\partial \xi} G(\xi;\kappa_{i,j+\frac{1}{2}},\kappa_{i+1,j+\frac{1}{2}}, M_{i+\frac{1}{2},j+\frac{1}{2}})\right|_{i,j+\frac{1}{2}}~,
    \end{aligned}
\end{equation} 
%the interpolations rely both on 
where the surface integral $M_{i+\frac{1}{2},j+\frac{1}{2}}=\int_{\xi_i}^{\xi_{i+1}}\int_{\Omega_j}^{\Omega_{j+1}}g(t,\xi,\Omega)d\xi d\Omega$.
% and it can be readily transformed to the loop integral 
The time evolution of this surface integral according to the Stokes theorem is given by
\begin{equation}\label{eq.area}
    \begin{aligned}
         \left.\frac{\partial M}{\partial t}\right|_{i+\frac{1}{2},j+\frac{1}{2}}
         &= \int_{\xi_i}^{\xi_{i+1}} n_{j+1}(\xi) g_{j+1}(\xi)~\mathrm{d}\xi - \int_{\Omega_j}^{\Omega_{j+1}} m_{i+1}(\Omega) g_{i+1} (\Omega)~\mathrm{d} \Omega \\
        & - \int_{\xi_i}^{\xi_{i+1}} n_{j}(\xi) g_{j}(\xi)~\mathrm{d}\xi + \int_{\Omega_j}^{\Omega_{j+1}} m_{i}(\Omega) g_{i} (\Omega) \mathrm{d}\Omega \\
        &+ \iint \mathcal{S}(\xi,\Omega)~\mathrm{d}\xi\mathrm{d}\Omega~.
    \end{aligned}
\end{equation}
%The line interals in above equation are then solved similarly as in Eq~(\ref{eq.line}), and the area 
Here the integral of the source term is solved by trapezoidal integration.
%The RK method
Equations (\ref{eq.disV}), (\ref{eq.line}), and (\ref{eq.area})
are a set of ordinary differential equations that can be solved  by Runge-Kutta (RK) method.
%-------------------------------%-------------------------------
%Boundary conditions
For the boundary conditions, 
%within one Lagrangian marker, the spatial grid acctually extents over several wave lengths and should contains various $F(t,\xi,\Omega)$.  When staying the resonant frame, the translation motion of the resonances is shifted to zero and the resonances sit within the reference frame. The resonanct particles are regarded as identical, so the boundary condition of 
the distribution function 
%between two neighbor resonances 
in  the resonant frame
is assumed to be periodic in the phase angle $\xi$,
\begin{equation}
    g\left(\xi,\Omega,t\right)=g\left(\xi+2 \pi,\Omega,t\right)
\end{equation}
and
%Besides, for the delta $f$ method, we manually set the momentum for the perturbed distribution, thus 
the values of the perturbed distribution vanish at the $\Omega$ boundaries.
% is simply vanished, i.e.,
% \begin{equation}
%     g\left(\Omega\right)=0~.
% \end{equation}
%Nota that, in our simulation the calculation frame is shifted to the approximate resonant frame, which uses the linear frequency and the mode number determined by the linear dispersion relation to compute the resonant velocity with no chirping. Once the frequency chirping occurs, the actual resonant velocity deviates from the calculation frame and advances the resonance in the middle of the cell eventually to move outside it, which causes the breakdown of the periodic boundary condition. During the frequency chirping, the resonance still needs to spend a finite time to move through the cell with the spreading $l$ in the approximate resonant frame, which is the time window to validate our simulation. The time window can be extended as long as the resonances are frozen in the approximate resonant frame, which is as close as the actual resonant frame. 
%For the better unstand the scheme, we breifly illustrate the simulation domain and the principle procedures of the hybrid scheme in Fig.~\ref{fig.demo}. The above numerical solution for local distribution is refereed as process A in Fig.~\ref{fig.demo}.
\begin{figure}[htbp]
    \centering
    \includegraphics[scale=0.5]{cpc_img/Hybrid_demo.pdf}
    \caption{Schematic illustration of the hybrid Vlasov simulation scheme.}
    \label{fig.demo}
\end{figure}

\subsection{The Lagrangian step}
%After evolving the perturbed distribution on the $\xi-\Omega$ grids, 
%to solve the rest part of the Vlasov equation (\ref{eq.deltaf}) with respect to $s_i, \mathcal{J}$,
For the Lagrangian markers,
%with label $l$ for the coordinate $ \mathcal{J}$, 
%without the calculation for $\vartheta$, we sample the markers at $s$ and $\mathcal{J}$ dimension.
% We use $N_s$ points for sampling $s$, and use $N_\mathcal{J}$ sampling points in $\mathcal{J}$ dimension for each $s_i$. 
% Thus, the binary index $(i,j)$ is used to denote the marker, and the total number of the Lagrangian markers is $N_s \times N_\mathcal{J}$.
% On the $s$ dimension, $N_s$ is equal to $N_g$, the number of the grids used to discretize the wave equation, and the markers are initially set one to one onto the fixed wave grids $s_g$.
%Sampling of J
the sampling point for $\mathcal{J}$ is chosen to ensure  a uniform sampling to the initial equilibrium distribution,
%is used to sample the $\mathcal{J}$ coordinate
%function,
% at a given $s$, 
\begin{equation}
    f_0(\mathcal{J}) = l/N_l, 
    %~ j = 1, N_\mathcal{J}~,
\end{equation}
where $N_l$ is the total number of sampling points for $\mathcal{J}$. An illustration is shown in Fig. \ref{fig.uni_grid}(a).

%Since the resonant velocity varies along the resonance frame coordinate $s$,  
%to persist the spatial alignment between Lagrangian markers and the fixed grid, we apply a nonuniform grid i.e., an initial nonuniform sampling points.To arrange the initial
For the spatial sampling of the marker, we put them initially on the fixed grid, $s_i$, the grid used to solve the wave.
To ensure that time for every marker goes through one grid is identical, we also apply nonuniform sampling, thus nonuniform wave grid.
We first calculate the transit time of a marker through the whole simulation domain
%from $s_1$ and $s_{N_g}$ denoting the two end points of the domain,
\begin{equation}
    T = \int_{s_1}^{s_N} \frac{\mathrm{d}s}{v_r(s)}~
\end{equation}
where $v_r(s)$ is the local resonant velocity that has been predetermined from the background equilibrium parameter.
Then the simulation time step is set as $\Delta t = T/N$ with $N$ the total number of sampling points/grids.
Finally the initial spatial coordinates of the markers are set according to equation (\ref{eq.resonance}) gives the trajectory of the Lagrangian marker. 
\begin{equation}
    s_{k+1} = s_k +  \int_{t}^{t+\Delta t} v_r(s_k(t)) \mathrm{d}t~
\end{equation}
where $k=1,2,3,...,N$.
The nonuniform grids with size $\Delta s$ are shown in Fig. \ref{fig.uni_grid}(b). 
%An illustration of the nonuniform is 
With such nonuniform grid, the initially sampled Lagrangian markers on the grid reach the next adjacent grid after a time interval $\Delta t$.
\begin{figure}[htbp]
    \centering
    \includegraphics[scale=0.5]{cpc_img/fig_nu_grid.pdf}
    \caption{The nonuniform spatial grid obtained from resonance velocity. 
    %Here the number of grid points $N=500$.
    }
    \label{fig.uni_grid}
\end{figure}
%where $t_i = i \Delta t$.
%Push feon to new Lagrangian markers
%The trajectory for each Lagrangian marker $s_i,\mathcal{J}_j$ is determined by equations (\ref{eq.resonance}) and (\ref{eq.Lagrangian}).
%Thus the distribution $g_{k,l}(t^{n+1},\xi,\Omega)$ from the Eulerian step is pushed to the new location $s_{k+1},\mathcal{J}_l$ of the Lagrangian marker
Thus we successively push the distribution $g_{k,l}(\xi,\Omega)$  from  $s_k$ to the next grid $s_{k+1}$ for each time interval $\Delta t$.
%drop the maker which leaves the simulation zone, and set a new distribution $g(\xi,\Omega) = 0$ for which enters the simulation domain at each time iteration. In the meantime, $\mathcal{J}_j$ is updated by the Runge-Kutta (RK) method.
%, as shown in $B.2$ route. 
After pushing the Lagrangian maker from the $n^\mathrm{th}$ to the $n+1^\mathrm{th}$ time step, 
%the distribution $g(t^{n+1},\xi,\Omega)$ from the Eulerian iteration is moved to this new location $s_i^{n+1},\mathcal{J}^{n+1}_j$. 
the Hamiltonian  is re-calculated by  the  equilibrium quantities  at the new location of $s$ and $\mathcal{J}$, 
%$s_i^{n+1},\mathcal{J}^{n+1}_j$ 
which are needed for evolving the distribution in $\xi-\Omega$ space at the next time step.

We also apply the uniform grid and sampling of Lagrangian marker on $s$ to verify the nonuniform grid approach. 
After a given time step $\Delta t$ the markers deviated from the fixed grid, thus  we need to interpolate the distribution function and its derivatives required in the IDO scheme from the adjacent marker to the fixed $s$ grid.
The procedures are similar to the semi-Lagrangian method \cite{sonnendrucker1999,cottet2018}.
After interpolation, the markers realigned to the grid, and the next iteration start again from the grid location.
As shown in Fig.~\ref{fig.cmp1}, the wave amplitudes calculated from  the two approaches converge  as the smaller grid sizes are used for  the   uniform grid with interpolation.  
\begin{figure}[htbp]
    \centering
    \includegraphics[scale=0.5]{cpc_img/fig_semiL.pdf}
    \caption{Wave amplitude from semi-Lagrangian method with linear interpolation of distribution at different uniform grid sizes (dashed-line) and the nonuniform grid (solid line).
    }
    \label{fig.cmp1}
\end{figure}
The evolution of $\vartheta,\mathcal{J}$ in Eq. (\ref{eq.Lagrangian}) are simply solved by applying the RK method.

Since the evolution in $\xi,\Omega$ phase space is much faster than that in the $s_i,\vartheta,\mathcal{J}$ domain,
we use the large fixed time step $\Delta t$ for the Lagrangian calculation
and  apply a much smaller time step $\Delta t_{adp}$ for solving the fast varying dynamics on the Eulerian domain. 
% Here, we apply adaptive time step RK method to solve the dynamcis within one $\Delta t$, i.e., 
Here the smaller time steps $\Delta t_{adp}$ are adaptively obtained from real time error analysis using the  adaptive time step RK method.
%of the second and the third order RK during step $A$ as shown in Fig.~\ref{fig.demo}.
Thus 
%As the time step adaptively adjust its size, within one $\Delta t$ is also dynamical, and satisfies
\begin{equation}
    \sum_{n=1}^N \Delta t_{adp}= \Delta t ~.
\end{equation}
where $N$ is the total number of iterations within one $\Delta t$. 
%The last fast scale time difference $\Delta t_{adp}$ at $N-1$ th iteraction is given by 
% \begin{equation}
%     \Delta t_{adp}^{N-1} = \Delta t - \sum_0^{N-1} \Delta t_{adp}
% \end{equation}
%We intutively visualize the time step varying during the iteraction 
As shown in Fig.~\ref{fig.adapetive}, 
%At each step within one $\Delta T$, 
$\Delta t_{adp}$ is adjusted and reduced to a smaller value to satisfy the error constraint. As the system evolves nonlinearly, 
the  time step $\Delta t_{adp}$ is also refined and the iterations automatically increase within one $\Delta t$. It is clearly that the adaptive hybrid methods greatly enhance the computation efficiency.
\begin{figure}[htbp]
    \centering
    \includegraphics[scale=0.5]{cpc_img/fig_dts1.pdf}
    \caption{The variation of time step $\Delta t_{adp}$ 
     with the number of iterations for Eulerian calculation.  
    %choosen adaptively in different times of the Lagrangian grid. For each Lagrangian-Eulerian time step, hundreds of interactions are carried on Eulerian grid, and the time step is converged to smaller values until reach $\Delta T$. As the system evolves nonlinearly, the iteractions within one step is increasing due to smaller $\Delta t_{adp}$
    }
    \label{fig.adapetive}
\end{figure}


%then we sove the following envelope equation 
%explain the form
%\input{cpc/methods}
\section{Wave Equation}
\label{sec:wave}

We proceed to integrate the HEL Vlasov solver with the wave equation, enabling us to self-consistently investigate resonant wave-particle interactions. Here we focus on the interaction between energetic electrons and chorus waves in the Earth's magnetosphere. Chorus waves are circularly polarized electromagnetic waves that propagate along the Earth's dipole magnetic field. The vector potential A(s, t) representing the transverse wave is expressed as follows:
\begin{equation}
    \mathbf{A}(s,t) = \mathbf{a}(s,t)e^{\imath \phi_f}~,
\end{equation}
where $\phi_f$ is the fast varying phase and $\mathbf{a}(s,t)$ is the slowly varying envelope of the wave packet. 
The normalized second-order wave equation for the chorus in the resonance frame  is 
\begin{equation}\label{eq.Wave}
    \begin{aligned}
        &\frac{\partial^2 a}{\partial t^2} - \frac{\partial^2 a}{\partial s^2} + {2\imath\omega_l}\frac{\partial a}{\partial t} + 2\imath k_l\frac{\partial a}{\partial s} + \\
        &\frac{\omega_{pe}^2 \omega_{ce}}{(\omega_{ce}-\omega_l)} \int_0^t d \tau \frac{\partial a}{\partial \tau} e^{-\imath\left(\omega_l-\omega_{c e}\right)(t-\tau)} = j_p~,
        \end{aligned}
      \end{equation}
where $a = a_x + \imath a_y$ with $x$ and $y$ the directions perpendicular to the  magnetic field. Note that the most unstable frequency $\omega_l$ and wave number $k_l$ used in the simulation are obtained from  the linear whistler disperison relation.
The plasma frequency $\omega_{pe}(s)$ and electron gyrofrequency $\omega_{ce}(s)$
are determined by the background plasma density and magnetic field profile along the magnetic field line.
The plasma current is a combination of contributions from both cold electrons and energetic electrons. The cold electron current can be analytically integrated. 
The energetic electron current $j_p$ is obtained from the perpendicular velocity moment of energetic particle distribution,
\begin{equation}
    j_p(s,t) = - 
ek_l(t)
    %\frac{en_{h0}k_l(t)}{m_e}
    \iiint 
    %\sqrt{2m_e\omega_{ce}(s)(\mathcal{J}+\Omega+\Pi_i)}
%m_e  
v_\perp
    f(\xi,\Omega;\mathcal{J};s,t) e^{\imath \xi} \rm d \xi \rm d \Omega \rm d \mathcal{J}~.
\end{equation}

The second-order wave equation (\ref{eq.Wave}) can be expressed as a system of first-order ordinary differential equations,
\begin{equation}\label{eq.Wave2}
    \begin{aligned}
        \frac{d y_0}{d t} & = y_1 %+ D \frac{\partial^2 y_0}{\partial s^2}
        \\
        \frac{d y_1}{d t} & =-2 \imath \omega_l y_1+\frac{\partial^2 y_0}{\partial s^2}-2 \imath k_l \frac{\partial y_0}{\partial s}- \frac{\omega_{pe}^2\omega_{ce}}{\omega_{ce}-\omega_l}y_{2} +j_p\\
        \frac{d y_2}{d t} & =y_1-\imath\left(\omega_l-\omega_{ce}\right) y_2~,
        \end{aligned}
\end{equation}
where 
 %The terms $y_0,y_1$ and $y_2$ are
\begin{equation}
    \begin{aligned}
        y_0 &= a(s,t)~,
        \\
        y_1 &= \frac{\partial a(s,t)}{\partial t}~,
        \\
        y_2 &= e^{-\imath(\omega_l-\omega_{ce})t} \int_0^t\mathrm{d}\tau \frac{\partial a}{\partial \tau} e^{\imath(\omega_l-\omega_{ce})\tau}~.
    \end{aligned}
\end{equation}
%Note that we have added a numerical diffusive term with $D$ the diffusive coefficient in the first equation.
The grid for solving the wave equation is the same as the grid for Lagrangian markers.
For the nonuniform grid, 
%to ensure the order of precision, 
a linear combination of finite difference scheme is employed and the first and second order spatial derivatives are given by 
\begin{equation}
    \frac{\partial f}{\partial z} \approx \frac{f(z_k) - f(z_{k-1})}{z_k-z_{k-1}} + \frac{f(z_k) - f(z_{k+1})}{z_k - z_{k+1}} - \frac{f(z_{k+1}) - f(z_{k-1})}{z_{k+1} - z_{k-1}}~,
\end{equation}
and
\begin{equation}
    \frac{\partial^2 f}{\partial z^2} \approx 2 \left(\frac{f(z_{k-1})}{(z_{k-1}-z_{k})(z_{k-1}-z_{k+1})}+\frac{f(z_{k})}{(z_k-z_{k+1})(z_k-z_{k-1})}+\frac{f(z_{k+1})}{(z_{k+1}-z_{k})(z_{k+1}-z_{k-1})}\right)~.
\end{equation}
The ordinary differential equations (\ref{eq.Wave2}) 
%in terms of time are subsequently re
are then advanced in time using the RK method.

To check the contribution of the second order derivatives of slowly varying kernel $a(s,t)$ in the wave equation (\ref{eq.Wave}), we further simplify the integral term and obtain the first-order advective wave equation 
\begin{equation}\label{eq.Wave1st}
    \frac{\partial a}{\partial t} + v_{g} \frac{\partial a}{\partial s} = \frac{2\pi v_g}{k_l} j_{p}~,
\end{equation}
where $v_g(s)$ is the linear group velocity determined by the whistler disperison relation.
Using the implicit upwind scheme,
 the discretized form of Eq. (\ref{eq.Wave1st})  is
\begin{equation}
a_k^{n+1} = a_k^{n} - u_{k}\cdot\left(a_{k}^{n+1}-a_{k-1}^{n+1}\right)\frac{\Delta t}{\Delta s} + S_k^n \Delta t
   % a^{n+1}_k = a^{n}_k \pm {u_k} (a_{k\pm1}^{n+1} - a_{k}^{n+1})\frac{\Delta t}{\Delta s}+S_k^n \Delta t
    % \begin{aligned}
    % a^{n+1}_k &= \left[ S_k^n \pm  \frac{\mathrm{imp}}{\Delta s}  {u_k} a_{k\pm1}^{n+1} + \left(\frac{1}{\Delta t} - \frac{(1- \mathrm{imp})}{\Delta s} u_k\right)  a_{k}^{n} \pm  \frac{(1-\mathrm{imp})}{\Delta s} u_k a_{k \pm 1}^{n} \right]/\left(\frac{1}{\Delta s} + \frac{\mathrm{imp}}{\Delta s}u_k\right)
    % \end{aligned}
\end{equation}
where 
the subscript $k$ is the spatial grid index
and
the superscript $n$ is the time index.
Here 
$u_k $ denotes $ v_g(s=s_k)$ and 
%$\mathrm{imp}$ is the combination factor of the implicit-explicit upwind scheme. Here 
$S$ is the source term on the right-hand-side of Eq. (\ref{eq.Wave1st}).
%The sign of the advective term depends on the direction of 
For the wave packet traveling from left to right 
with a group velocity  $v_g>0$, 
we apply 
a fixed noisy initialization condition at the left end and 
absorption boundary conditions at the other end of the domain.
Figure \ref{fig.cmp2} demonstrates that the first-order wave equation provides a strong approximation to the second-order wave equation. Consequently, the second-order term in Eq. (\ref{eq.Wave}) may be neglected when simulating the onset of chorus wave frequency chirping for the sake of computational efficiency.
\begin{figure}[htbp]
    \centering
    \includegraphics[scale=0.5]{cpc_img/fig_diff.pdf}
    \caption{ Wave amplitudes calculated from the second-order and first-order wave equations during the onset of chorus wave frequency chirping. 
    }
    \label{fig.cmp2}
\end{figure}

%the details of the code 
%\input{cpc/results}
\section{Benchmark Results for the Onset of Rising Tone Chorus}
\label{sec:code}
We proceed with a series of simulations, employing our method to investigate the onset of the whistler-mode chorus
 in the  magnetosphere. The instability of the whistler wave typically arises from the electron temperature anisotropy  \cite{kennel1966a,kennel1966b} and the growth rate can be derived from the linear resonance condition.
 % near $\Omega \approx 0$.
In our simulation, the energetic electron distribution is bi-Maxwellian at the magnetic equator, 
%Moving away from the magnetic equator, the equilibrium distribution maintains the similar form as it does in the phase space by conventional coordinates $(s_i, p_\|, \phi, \varphi)$. The distribution function in new canonical variables is given directly by the canonical transformation in ref \cite{zheng2023a}, and we have 
\begin{equation}
    \begin{aligned}
        & f_{0}(\Omega, \mathcal{J}) =\frac{\omega_{c e 0}}{(2 \pi)^{3 / 2} v_{ \perp 0}^2 v_{ \| 0}} \cdot \exp \left(-\frac{k_l^2(\Omega+\Pi_i)^2}{2 v_{ \| 0}^2}\right) \\
        &\cdot\exp \left(-\frac{(\mathcal{J}+\Omega+\Pi_i) \omega_{c e 0}}{v_{ \perp 0}^2}\right)~,
        \end{aligned}
\end{equation}
where $v_{ \| 0}$
and $v_{ \perp 0}$ are the
%$\beta$ is the depth of the loss cone 
parallel and perpendicular thermal velocity of energetic electrons,
and 
the subscript $0$ denotes the magnetic equator.
%As a realistic model for, The Earth's magnetic field can be approximated as a dipole field and 
%the major component of 
The background magnetic field strength near the equator is approximated %represented
 by a parabolic function \cite{tao2014a}
\begin{equation}
    B(\lambda) = B_0 (1+ R_a \lambda ^2)~,
\end{equation}
where $R_a$ is the inhomogeneity parameter of the magnetic dipole field, $B_0$ is the magnetic field strength at the equator, and $\lambda$ is the magnetic latitude. %The distance  satisfies $s = L R_E \lambda$.
For the dipole field, 
%the magnetic strength decays as the cube of the radius increases, i.e.,
%\begin{equation}
 $   B_0 = B_{0g}/L^3$
%\end{equation}
where $B_{0g}\approx $0.3 G is the magnetic field of the Earth  at the equator 
%on the  surface of Earth 
and $L$ is the L-shell of the magnetic field line.
The background electron density along the magnetic field line  is  fit to a power law \cite{denton2004},
\begin{equation}
    n(\lambda) = n_0 (1+R_b \lambda^2)~,
\end{equation}
where $R_b$ represents the 
background cold plasma density inhomogeneity.
For the normalization in the simulation, 
 we use electron mass $m_e$, electron charge $e$, speed of light $c$ and background electron plasma frequency at the magnetic equator $\omega_{pe0}$ to normalize mass, charge, velocity and frequency.
Then the  units for the other quantities can be derived. 
For example,
the length $s $ is normalized to $ c/\omega_{pe0}$
and 
 the vector potential $A$ to $m_e c^2/e$.




   \begin{figure}[htbp]
        \centering
        \includegraphics[scale=0.5]{cpc_img/fig_profile.pdf}
        \caption{(a) Background magnetic field and density profile. (b) Resonant and linear group velocity profile along the magnetic field line.}
        \label{fig.profile}
    \end{figure}
\subsection{Simulation configuration}

\begin{table}\label{tab.parameters}
    \centering
    \caption{Magnetic field and plasma parameters used in the simulation.\newline}
    \begin{tabular}{lc}
    \hline
     L-shell of the magnetic field line  & 5 \\
     Magnetic field inhomogeneity  $R_a$ &  4.5 \\
     Background cold plasma density inhomogeneity  $R_b$ &  1.0 \\
     Background electron gyrofrequency and plasma frequency at the equator & \makecell{ $\omega_{ce0} = 43929.6~\mathrm{rad/s}$\\$\omega_{pe0} = 5 \omega_{ce0}$  }\\
     Ratio of energetic to cold  electron density at the  equator &  $n_{h0} = 0.002$ \\ 
     Parallel and perpendicular thermal velocity of energetic electrons & \makecell{$v_{\perp 0} = 0.3 c$\\ $v_{\|0} = 0.15c$}  \\
    %Depth of the loss cone $\beta$ & 0 \\
    Size of the simulation domain  & \makecell{$\lambda \in [-15^\circ, 15^\circ]$ \\ $s \in [-6115,6115] c/\omega_{pe0}$} \\
    \hline
    \end{tabular}\\
    \end{table}
%(useless) Note that, since we apply the electron plasma frequency $\omega_{pe0}$ at the equator for time normalization, the value always be 1 desites the L-shell value, which gyrofrequency is dynamcally changes.
 
In conventional PIC simulations, the inhomogeneity ratio $R_a$ is often increased by one or two order of magnitudes to reduce the simulation cost.
%In contrast,
%However, one of the advantages of 
% is that we can
Benefiting from the scale-separated HEL  scheme,
 the realistic parameters for the Earth's dipole field
can be used in our numerical simulation.
The basic parameters 
of our simulation 
are given in Table (\ref{tab.parameters})
and the profiles of the background parameter are shown in Fig. \ref{fig.profile}.
%Meanwhile, 
The most unstable wave frequency $\omega_l$ used in the simulation 
%for the %determination of initial reference frame
% According to the choosen parameter in Tab. (\ref{tab.parameters}) and the definition of
 is determined from the linear growth rate  $\gamma_l$ \cite{gary_1993},
\begin{equation}
\begin{aligned}
    \gamma_l(s) & =\frac{\sqrt{2 \pi} \omega_{ce} v_g n_{h0} \omega_{pe0}^2}{4 k_l^2 v_{ \| 0}} e^{-\frac{\left(\omega_l-\omega_{c e}\right)^2}{2 k_l^2 v_{ \| 0}}}
    \cdot \left( \frac{T_{\perp 0}}{T_{\| 0}} \frac{\omega_{c e 0}-\omega_l}{\omega_{c e 0}}-1\right)~,
    \end{aligned}
\end{equation}
where 
 $n_{h0}$ is the density ratio of the energetic electrons to the background cold plasma at the magnetic equator,
the wave number $k_l$ satisfies the linear whistler disperison relation
\begin{equation}
    \frac{c^2 k_l^2(s)}{\omega_l^2} = 1 + \frac{\omega_{pe}^2(s)}{\omega_l(\omega_{ce}(s)-\omega_l)},
\end{equation}
and $v_g$ is the linear whistler group velocity,
\begin{equation}
    v_g(s) = \frac{2k_l(s)c^2 }{2\omega_l + \omega_{pe}^2(s)\omega_{ce}(s)/(\omega_{ce}(s)-\omega_l)^2}~.
\end{equation}
The temperature is given by the thermal velocity shown in Tab. (\ref{tab.parameters}).
For the  parameters in Tab. (\ref{tab.parameters}), 
the most unstable 
growth rate is $\gamma_l \simeq 3.24\times 10^{-4}$
and the corresponding 
frequency is $\omega_l = 0.061$, as shown by a scan of the parameter given in Fig.~\ref{fig.para}.
For the typical runs, we set the number of grids for wave solver to be 1001
which is equal to  the number of sampling points in $s$.
In the $\mathcal{J}$ dimension, we employ a delta function with only a single sampling point for $\mathcal{J}$. In the case of a non-delta distribution, tens of sampling points prove to be sufficient for $\mathcal{J}$ sampling.
For the $\xi,\Omega$ domain, the Eulerian grids are $31 \times 401$ for $\xi \in [0,2\pi]$ and $\Omega \in [-0.1,0.1]$.
\begin{figure}[htbp]
    \centering
    \includegraphics[scale=0.5]{cpc_img/fig_gamma1d.pdf}
    \caption{Linear growth rate $\gamma_l$ with respect to $\omega_l$ for the simulation parameters. The initial frequency of the wave is obtained from the most unstable frequency indicated by the vertical black line.}
    \label{fig.para}
\end{figure}

\subsection{Benchmarks results}
The linear physics are quantitatively verified in Fig. \ref{fig.linear}, where we calculate the growth rate and the velocity of the maximum wave amplitude location before nonlinear effects become dominant.
 %In the case of the  wave packet, we determine
  The trajectory of the propagating wave packet is determined theoretically using the integral of the linear group velocity,
\begin{equation}
    s(t) = s(0) + \int_0^{t} v_g(s(\tau)) \mathrm{d} \tau~.
\end{equation}
For the simulation,
 we track the movement of the maximum amplitude point. 
%Its propagation is in accord with the linear group velocity, 
 As shown in Fig. \ref{fig.linear}(a), 
 the wave peak indeed propagates at the linear group velocity during the linear stage.
 The 
 amplitude growth of the wave peak along its propagation path
%growth of the wave peak  along the propatation path 
is shown in Fig. \ref{fig.linear}(b)
and 
%Moreover, we  
 the growth rate is estimated by \cite{nogi2022}
\begin{equation}\label{eq.gm_ver}
        \gamma = \frac{1}{t_1-t_0}\log\frac{|a(s_1,t_1)|}{|a(s_0,t_0)|}~,
\end{equation}
where $s_0$ and $s_1$ are the points along the propagation path.
The  growth rate calculated from Eq. (\ref{eq.gm_ver}) is $\gamma \simeq 3.21\times10^{-4}$, which agrees with the theoretical linear growth rate shown in Fig. \ref{fig.para}.
%The results show the correctness of our simulation at linear stage.
\begin{figure}[htbp]
    \centering
    \includegraphics[scale=0.5]{cpc_img/fig_linear.pdf}
    \caption{ (a) Propagation trajectory of the linear wave packet with the maximum amplitude from the simulation (dashed line) and theory (solid line).  (b)  Growth of the wave peak amplitude with time along wave propagation path. The green crosses are the points used to caluclate the growth rate.}
    \label{fig.linear}
\end{figure}

In the nonlinear stage, the distribution of trapped electrons forms a  hole structure in $\xi-\Omega$ phase space. This phase space hole contributes to the nonlinear current, which in turn triggers the nonlinear frequency-chirping chorus wave. 
% The Hamiltonian for the resonant electrons has the form %\cite{}
% \begin{equation}
% H = \frac{k^2\Omega^2}{2} + \frac{\omega_b^2}{k^2}\left(\cos \xi + \alpha \xi \right)~,
% \end{equation}
The trapped electron experiences  an osillatory force from the wave and a noninertial force from background inhomogeneity. The forces can be described by a potential well with the form $\sin \xi + \alpha \xi$ with the parameter $\alpha$ as an inhomogeneity ratio \cite{omura2008,tao2020}.
% \begin{equation}\label{eq.alp}
%     \alpha \equiv \frac{1}{\omega_{b}^2}\left[\left(1 - 2\frac{v_r}{v_g}\right)\frac{\partial \omega}{\partial t}  -v_r^2 \frac{\partial k}{\partial s}+ \frac{\mathrm{\partial} \omega_{ce}}{\mathrm{\partial} s}\frac{k_i}{m_e}\mathcal{J}\right].
% \end{equation}
From 
the inflection point of the potential well,
%the potential well, 
we can find the X point of the phase space trajectory, 
\begin{equation}
    \xi_\mathrm{x} = - \arcsin \alpha,
\end{equation}
and the C point of the trajectory is obtained from the energy on the separatrix $e_\mathrm{spx} = \sin \xi_\mathrm{x} + \alpha \xi_\mathrm{x}$,
\begin{equation}
    \sin \xi_\mathrm{c} + \alpha \xi_\mathrm{c} = e_\mathrm{spx}.
\end{equation}
The shape of the hole, specifically its boundary, can be analytically described as follows \cite{omura2008}:
\begin{equation}\label{eq.Omega_b}
    \Omega(\xi) = \pm \frac{\omega_b}{k^2} \sqrt{2 (e_\mathrm{spx}-\cos \xi - \alpha \xi)}~,
\end{equation}
where $k$ is the wave number and $\omega_b$
%\equiv \sqrt{k^2 v_\perp a}$ 
is the trapped particle bounce frequency.
Using
the simulation data we obtain the values of $k \simeq 0.649$, $\omega_b \simeq 0.007$, and $\alpha \simeq 0.08$ at a specific location $s\simeq 2107$. Then the boundary of the hole can be obtained using Eq. (\ref{eq.Omega_b}).
As shown 
in Fig.~\ref{fig.hole},
the shape of the hole in the nonlinear simulation
closely matches  the theoretical predictions.
%we present a typical electron phase-space hole observed at a specific location $s$ in the late nonlinear stage. We determine the corresponding values of $k \simeq 0.649$, $\omega_b \simeq 0.007$, $\alpha \simeq 0.08$ based on the simulation data at that location. Additionally, we illustrated the boundary of the hole using the equation (\ref{eq.Omega_b}). Notably, the shape of the hole closely aligns with the theoretical predictions.

\begin{figure}[htbp]
    \centering
    \includegraphics[scale=0.5]{cpc_img/fig_hole.pdf}
    \caption{Phase-space hole  formed by trapped electrons at $s\simeq 2107$ in the wave field. The X, C points on the left and right ends and the boundary of the hole (black solid line) are obtained from theory.}
    \label{fig.hole}
\end{figure}




\section{Conclusion}
\label{sec:end}
In summary, we have developed a hybrid Eulerian-Lagrangian method tailored for the resonance tracking Vlasov system
for the understanding of nonlinear resonant wave-particle interactions in inhomogeneous magnetic fields. This method effectively separates the perturbed wave-particle interactions within the fast varying Hamiltonian phase space while  advancing Lagrangian markers within the slowly varying Hamiltonian phase space. To validate the method, we have conducted self-consistent simulations focusing on the whistler-mode chorus in the Earth's magnetosphere  using realistic parameters. The results not only exhibit  agreement with the linear prediction  but also reveal high-resolution phase space structures of resonant energetic particles.
%The implications of this research are profound, spanning a spectrum of scientific disciplines, including fusion research, space physics, and astrophysics. This advancement in our understanding of nonlinear wave-particle interactions within inhomogeneous magnetic fields holds the potential to revolutionize our capacity to control and manipulate plasma behavior. This, in turn, can pave the way for progress in energy generation, propulsion, and space exploration. Thus, this paper represents a pivotal milestone in unlocking the untapped potential of nonlinear plasma physics within complex magnetic field environments.

%\input{cpc/appendix}

\bibliographystyle{model1-num-names}
\bibliography{nonlinear_cpc}

\end{document}

%%
