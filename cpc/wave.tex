\section{Wave Equation}
\label{sec:wave}

We proceed to integrate the HEL Vlasov solver with the wave equation, enabling us to self-consistently investigate resonant wave-particle interactions. Here we focus on the interaction between energetic electrons and chorus waves in the Earth's magnetosphere. Chorus waves are circularly polarized electromagnetic waves that propagate along the Earth's dipole magnetic field. The vector potential A(s, t) representing the transverse wave is expressed as follows:
\begin{equation}
    \mathbf{A}(s,t) = \mathbf{a}(s,t)e^{\imath \phi_f}~,
\end{equation}
where $\phi_f$ is the fast varying phase and $\mathbf{a}(s,t)$ is the slowly varying envelope of the wave packet. 
The normalized second-order wave equation for the chorus in the resonance frame  is 
\begin{equation}\label{eq.Wave}
    \begin{aligned}
        &\frac{\partial^2 a}{\partial t^2} - \frac{\partial^2 a}{\partial s^2} + {2\imath\omega_l}\frac{\partial a}{\partial t} + 2\imath k_l\frac{\partial a}{\partial s} + \\
        &\frac{\omega_{pe}^2 \omega_{ce}}{(\omega_{ce}-\omega_l)} \int_0^t d \tau \frac{\partial a}{\partial \tau} e^{-\imath\left(\omega_l-\omega_{c e}\right)(t-\tau)} = j_p~,
        \end{aligned}
      \end{equation}
where $a = a_x + \imath a_y$ with $x$ and $y$ the directions perpendicular to the  magnetic field. Note that the most unstable frequency $\omega_l$ and wave number $k_l$ used in the simulation are obtained from  the linear whistler disperison relation.
The plasma frequency $\omega_{pe}(s)$ and electron gyrofrequency $\omega_{ce}(s)$
are determined by the background plasma density and magnetic field profile along the magnetic field line.
The plasma current is a combination of contributions from both cold electrons and energetic electrons. The cold electron current can be analytically integrated. 
The energetic electron current $j_p$ is obtained from the perpendicular velocity moment of energetic particle distribution,
\begin{equation}
    j_p(s,t) = - 
ek_l(t)
    %\frac{en_{h0}k_l(t)}{m_e}
    \iiint 
    %\sqrt{2m_e\omega_{ce}(s)(\mathcal{J}+\Omega+\Pi_i)}
%m_e  
v_\perp
    f(\xi,\Omega;\mathcal{J};s,t) e^{\imath \xi} \rm d \xi \rm d \Omega \rm d \mathcal{J}~.
\end{equation}

The second-order wave equation (\ref{eq.Wave}) can be expressed as a system of first-order ordinary differential equations,
\begin{equation}\label{eq.Wave2}
    \begin{aligned}
        \frac{d y_0}{d t} & = y_1 %+ D \frac{\partial^2 y_0}{\partial s^2}
        \\
        \frac{d y_1}{d t} & =-2 \imath \omega_l y_1+\frac{\partial^2 y_0}{\partial s^2}-2 \imath k_l \frac{\partial y_0}{\partial s}- \frac{\omega_{pe}^2\omega_{ce}}{\omega_{ce}-\omega_l}y_{2} +j_p\\
        \frac{d y_2}{d t} & =y_1-\imath\left(\omega_l-\omega_{ce}\right) y_2~,
        \end{aligned}
\end{equation}
where 
 %The terms $y_0,y_1$ and $y_2$ are
\begin{equation}
    \begin{aligned}
        y_0 &= a(s,t)~,
        \\
        y_1 &= \frac{\partial a(s,t)}{\partial t}~,
        \\
        y_2 &= e^{-\imath(\omega_l-\omega_{ce})t} \int_0^t\mathrm{d}\tau \frac{\partial a}{\partial \tau} e^{\imath(\omega_l-\omega_{ce})\tau}~.
    \end{aligned}
\end{equation}
%Note that we have added a numerical diffusive term with $D$ the diffusive coefficient in the first equation.
The grid for solving the wave equation is the same as the grid for Lagrangian markers.
For the nonuniform grid, 
%to ensure the order of precision, 
a linear combination of finite difference scheme is employed and the first and second order spatial derivatives are given by 
\begin{equation}
    \frac{\partial f}{\partial z} \approx \frac{f(z_k) - f(z_{k-1})}{z_k-z_{k-1}} + \frac{f(z_k) - f(z_{k+1})}{z_k - z_{k+1}} - \frac{f(z_{k+1}) - f(z_{k-1})}{z_{k+1} - z_{k-1}}~,
\end{equation}
and
\begin{equation}
    \frac{\partial^2 f}{\partial z^2} \approx 2 \left(\frac{f(z_{k-1})}{(z_{k-1}-z_{k})(z_{k-1}-z_{k+1})}+\frac{f(z_{k})}{(z_k-z_{k+1})(z_k-z_{k-1})}+\frac{f(z_{k+1})}{(z_{k+1}-z_{k})(z_{k+1}-z_{k-1})}\right)~.
\end{equation}
The ordinary differential equations (\ref{eq.Wave2}) 
%in terms of time are subsequently re
are then advanced in time using the RK method.

To check the contribution of the second order derivatives of slowly varying kernel $a(s,t)$ in the wave equation (\ref{eq.Wave}), we further simplify the integral term and obtain the first-order advective wave equation 
\begin{equation}\label{eq.Wave1st}
    \frac{\partial a}{\partial t} + v_{g} \frac{\partial a}{\partial s} = \frac{2\pi v_g}{k_l} j_{p}~,
\end{equation}
where $v_g(s)$ is the linear group velocity determined by the whistler disperison relation.
Using the implicit upwind scheme,
 the discretized form of Eq. (\ref{eq.Wave1st})  is
\begin{equation}
a_k^{n+1} = a_k^{n} - u_{k}\cdot\left(a_{k}^{n+1}-a_{k-1}^{n+1}\right)\frac{\Delta t}{\Delta s} + S_k^n \Delta t
   % a^{n+1}_k = a^{n}_k \pm {u_k} (a_{k\pm1}^{n+1} - a_{k}^{n+1})\frac{\Delta t}{\Delta s}+S_k^n \Delta t
    % \begin{aligned}
    % a^{n+1}_k &= \left[ S_k^n \pm  \frac{\mathrm{imp}}{\Delta s}  {u_k} a_{k\pm1}^{n+1} + \left(\frac{1}{\Delta t} - \frac{(1- \mathrm{imp})}{\Delta s} u_k\right)  a_{k}^{n} \pm  \frac{(1-\mathrm{imp})}{\Delta s} u_k a_{k \pm 1}^{n} \right]/\left(\frac{1}{\Delta s} + \frac{\mathrm{imp}}{\Delta s}u_k\right)
    % \end{aligned}
\end{equation}
where 
the subscript $k$ is the spatial grid index
and
the superscript $n$ is the time index.
Here 
$u_k $ denotes $ v_g(s=s_k)$ and 
%$\mathrm{imp}$ is the combination factor of the implicit-explicit upwind scheme. Here 
$S$ is the source term on the right-hand-side of Eq. (\ref{eq.Wave1st}).
%The sign of the advective term depends on the direction of 
For the wave packet traveling from left to right 
with a group velocity  $v_g>0$, 
we apply 
a fixed noisy initialization condition at the left end and 
absorption boundary conditions at the other end of the domain.
Figure \ref{fig.cmp2} demonstrates that the first-order wave equation provides a strong approximation to the second-order wave equation. Consequently, the second-order term in Eq. (\ref{eq.Wave}) may be neglected when simulating the onset of chorus wave frequency chirping for the sake of computational efficiency.
\begin{figure}[htbp]
    \centering
    \includegraphics[scale=0.5]{cpc_img/fig_diff.pdf}
    \caption{ Wave amplitudes calculated from the second-order and first-order wave equations during the onset of chorus wave frequency chirping. 
    }
    \label{fig.cmp2}
\end{figure}